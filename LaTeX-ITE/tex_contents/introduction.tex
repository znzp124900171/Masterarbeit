\chapter{Introduction}

\section{Motivation}

Data visualization is an important method to demonstrate the results of numerical simulations. With graphics measures. the results can be presented through intuitive form and convey the information more effective, especially in the ares of teaching, product or research presentations, an easily available and lightweight visualization solution is expected\citeRefs{itep79}. Nowadays, with the development of web technologies and modern browsers, a cross-platform, web-based technologies WebGL(Web Graphics Library) make it possible.

Generally, people render 3D graphics by using particular APIs, e.g. DirectX, OpenGL. These APIs define a set of functions which can be called by the client program\citeRefs{wiki:opengl}. The solutions which build on these APIs need specific platforms, e.g. DirectX runs on Microsoft products, or specific configuration environment and plug-ins to deploy. These applications are also too heavy to run on mobile devices and their availability and compatibility are limited. The new web standard HTML5 brings up the new spark for web-based application. In general, HTML5 mainly involves the Hypertext Markup Language(HTML), the Cascading Style Sheets(CSS) and Javascript technologies, it introduces new features and new elements to enhance the support for multimedia and graphic, on the one hand the new specification reduces the dependencies on plug-ins, such as Flash, Silverlight, JavaFX, on the other hand it provides native support for graphics rendering with canvas element and JavaScript.

WebGL specification is official introduced into the HTML5 specification in 2014 as a web standard. It allows GPU acceleration and is supported by common browsers, such as Microsoft IE, Google Chrome, Apple Safari, Mozilla Firefox.

WebGL makes it possible for building an easy accessible, lightweight, cross-platform mobile visualization system.  The initial idea was came up in the paper "Web Based 3D Visualization for COMSOL Multiphysics®"\citeRefs{itep79}.

In this paper some further improvements are worked out and new features are added.

\section{Outline of this thesis}

This section outlines the chapters of this thesis.

\paragraph{Chapter 2: Visualization System}

\paragraph{Chapter 3: Web client}

\paragraph{Chapter 4: Rendering using WebGL}

\paragraph{Chapter 5: Virtual Reality}